\documentclass{article}

\usepackage{biblatex}

\addbibresource{ref.bib}

\title{ELE 548: Project Plan}

\author{Calvin Higgins}

\begin{document}

\maketitle

\section{Potential Topics}

\subsection{Machine Learning for Code Optimization}

\textbf{CompilerGym}~\cite{cummins2022} is a set of environments for compiler optimization tasks such as reducing code size and instruction count. To solve 
these tasks, agents select and apply pre-defined optimization passes to benchmark programs (i.e. agents try to solve the \textbf{phase-ordering problem}). 
Although several standard benchmark suites are supported, the \textbf{SPEC CPU2017 suite}~\cite{bucek2018} is notably absent. Moreover, CompilerGym only 
supports directly optimizing benchmark execution time on the \textbf{Jotai suite}~\cite{kind2025}. \\

\noindent
To address these limitations, I propose the following:
\begin{enumerate}
    \item Add the SPEC CPU2017 benchmark suite to CompilerGym, including code-size, instruction count and execution time observation spaces.
    \item Implement baseline methods (e.g. random search) for optimizing SPEC CPU2017 benchmark runtimes.
\end{enumerate}

\subsubsection{Performance and World Models}

In the absence of confounds such as operating system context switches, execution time is determined by three factors:
\begin{enumerate}
    \item Generated machine code.
    \item Microarchitecture implementation.
    \item Input data distribution.
\end{enumerate}

\noindent
Finding near-optimal sequences of passes (a) with few executions requires accurate \textbf{performance models} to estimate execution time, and (b) with few 
compilations requires accurate \textbf{world models} to approximate the impact of optimizations on either code representation or execution time. To be accurate, 
performance and world models must incorporate all three factors influencing execution time. \\ 

\noindent
However, existing methods for ordering optimization passes, such as \textbf{CompilerDream}~\cite{chaoyi2025} the only model-based RL method for phase-ordering, 
have primarily focused on code-size reduction. Some methods have addressed execution time, including \cite{liu2025} and \cite{patabandi2023}, although they have
focused on code, and not microarchitecture and data distribution representations. \\

\noindent
To address this limitation, I propose the following:
\begin{enumerate}
    \item Explicitly model the microarchitecture and data distribution, and inject these representations into an existing phase-ordering approach.
\end{enumerate}

\noindent
Alternatively, the combination of code, microarchitecture, and data representations could be evaluated via SPEC CPU2017 score prediction. Existing approaches 
are simple (e.g. multilayer perceptrons~\cite{ashkan2022}, convolutional neural networks~\cite{mehmet2023}), and only explicitly incorporate microarchitecture 
information. \\

\noindent
To address this limitation, I propose the following:
\begin{enumerate}
    \item Incorporate code and data representations into SPEC CPU2017 score prediction.
\end{enumerate}

\printbibliography

% survey paper: https://dl.acm.org/doi/pdf/10.1145/3197978?casa_token=-DDX5GUdAXoAAAAA:T2rg1-lxhWKAvF0o6LmjCqe4xmM_-JrdH5EjDbTANo9WqlPdpZEJCSNgguREu1zbFkbMrwEq_irI
% architecture survey paper: https://dl.acm.org/doi/pdf/10.1145/3494523?casa_token=jvcb3B5pe9QAAAAA:f2v9warN-BIT0uLMSI0UC0skUiJOKioEKHUyY4rZoMpyV5-E31nV80RjdD3NUMKZfelu1Tzg1shV

% DL models:
% https://dl.acm.org/doi/10.1145/3640537.3641582
% RL + LLM: https://hpcrl.github.io/ICS2025-webpage/program/Proceedings_ICS25/ics25-26.pdf

% non-DL models:
% https://dl.acm.org/doi/pdf/10.1145/3715756
% https://dl.acm.org/doi/10.1145/3735452.3735530

% more flexible than opt passes: https://dl.acm.org/doi/pdf/10.1145/3696443.3708922
% https://arxiv.org/pdf/2410.03120v2
% register allocation: https://dl.acm.org/doi/pdf/10.1145/3578360.3580273

% compilergym for architecture: “ArchGym: An open-source gymnasium for machine learning assisted architecture design,
% compilergym imporvement: https://dl.acm.org/doi/pdf/10.1145/3640537.3641580

% vision paper (cummins): https://scontent-bos5-1.xx.fbcdn.net/v/t39.8562-6/240841093_348724053629846_1551966423153256824_n.pdf?_nc_cat=104&ccb=1-7&_nc_sid=e280be&_nc_ohc=SBNzHCt9NUoQ7kNvwHMzHwR&_nc_oc=AdmwqWSAXj0GwWozUgvfgjldeKBXzA0ZPg1Pytsa3n1GEdBsWjMZ968aHmlnh9MfeStWISh05txE9_6FIqqjqNCP&_nc_zt=14&_nc_ht=scontent-bos5-1.xx&_nc_gid=rvUDSW0JDs3fG_3T1T4RRg&oh=00_AfcfOmoYuZ7ggtc5oyq7FQyeOq7mT-cDF7abi7UNmOX2lg&oe=68F5FD6B

% find heuristics in code, optimize them: https://dl.acm.org/doi/10.1109/CGO57630.2024.10444847

% polyhedral: https://dl.acm.org/doi/10.1145/3721145.3725766

% find performance bugs in compilers by running LLM code: https://dl.acm.org/doi/pdf/10.1145/3708493.3712686

\end{document}
